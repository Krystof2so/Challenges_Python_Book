% CHAPITRE 1
\chapter{Compter le nombre de voyelles}
\vspace{2cm}
\section{Énoncé}
\medskip

Ce premier challenge est très simple, il est de niveau \og débutant\fg{}, mais si vous avez plus d'expérience, vous pouvez essayer de trouver de belles astuces pour un code propre, rapide et concis.
\medskip

Ici, il va nous falloir créer une fonction \verb|nb_voyelles(phrase: str)->int| qui retourne le résultat du nombre total de voyelles dans une phrase passée en paramètre.
\medskip

\subsection*{Conditions}
\begin{itemize}
	\item[-] Les voyelles sont : \texttt{aeiou}, \texttt{y} n’est pas pris en compte.
	\item[-] Les voyelles accentuées ne sont pas prises en compte.
	\item[-] La phrase passée en paramètre doit être écrite en minuscule.
	\item[-] Une chaîne vide, passée en paramètre, doit renvoyer \texttt{0}.
\end{itemize}
\medskip

\subsection*{Exemples}
\begin{itemize}
	\item[\textbullet] \verb|nb_voyelles("bonjour, comment allez-vous ?")| doit retourner \texttt{9}.
	\item[\textbullet] \verb|nb_voyelles("je vais à paris")| doit retourner \texttt{5}.
	\item[\textbullet] \verb|nb_voyelles("docstring")| doit retournet \texttt{2}.
	\item[\textbullet] \verb|nb_voyelles("")| doit retourner \texttt{0}.
\end{itemize}
\medskip

\section{Solution et explications}
\medskip

Voici donc ma solution\footnote{Fil de discussion de ce challenge: \url{https://discord.com/channels/396825382009044994/1142617945139335189}} :
\begin{lstlisting}
def nb_voyelles(phrase: str)->int:
	return sum(phrase.count(el) for el in "aeiou")
\end{lstlisting}
\medskip

\begin{itemize}
	\item[\textbullet] La phrase doit toujours être en minuscule, donc pas besoin de la méthode \texttt{lower()}.
	\item[\textbullet] Ici, on compte chaque voyelle dans la phrase à l'aide de la méthode \texttt{count}.
	\item[\textbullet] La fonction \texttt{sum()} renvoie ensuite la somme du résultat obtenu.
	\item[\textbullet] Si l'on passe un générateur ou une liste de compréhension dans la fonction \texttt{sum()}, la paire de crochets supplémentaire peut-être éliminée\footnote{ Attention, car cela n'est par contre pas compatible avec la fonction \texttt{len()}. Pour plus d'information on se reportera au \texttt{PEP-289}: \url{https://peps.python.org/pep-0289/\#the-details}.}. De cette manière :
	\begin{verbatim}
	sum([phrase.count(el) for el in "aeiou"])
	\end{verbatim}
	est l'équivalent de :
	\begin{verbatim}
	sum(phrase.count(el) for el in "aeiou")
	\end{verbatim}
\end{itemize}
\medskip

Voici aussi le code pour mes tests unitaires :
\begin{lstlisting}
import pytest

@pytest.mark.parametrize("sentence, expected", [
    ("", 0),
    ("docstring", 2),
    ("bonjour comment allez-vous ?", 9),
    ("je vais à paris", 5),
    ("vas-y !", 1),
]) 
def test_should_return_the_sum(sentence, expected):
    got = nb_voyelles(sentence)
    assert got == expected
\end{lstlisting}
\medskip

\texttt{@OsKaR31415} a par ailleurs apporté plusieurs solutions pour résoudre ce challenge\footnote{\url{https://discord.com/channels/396825382009044994/1142617945139335189/1144591818516860998}}.