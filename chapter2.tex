\chapter{Jeu du \og Pierre - Papier - Ciseaux\fg{}}
\vspace{2cm}
\section{Énoncé}
\medskip

On va jouer un peu en développant un petit jeu très simple.
\medskip

Le but du challenge est de développer le célèbre jeu \og \textit{pierre - papier - ciseaux}\fg{}\footnote{\url{https://fr.wikipedia.org/wiki/Pierre-papier-ciseaux}} en essayant de trouver un algorithme astucieux et un code à la fois simple, propre et efficace.
\medskip

\subsection*{Étapes}
\begin{enumerate}
	\item Générer un choix aléatoire pour votre session de jeu : \og pierre\fg{}, \og papier\fg{} ou \og ciseaux\fg{}.
	\item Demander au joueur d'écrire son choix entre trois propositions : \og pierre\fg{}, \og papier\fg{} ou \og ciseaux\fg{}.
	\item Afficher qui a gagné en dévoilant le choix aléatoire du point n°1.
\end{enumerate}
\medskip

\subsubsection*{Conditions}
\begin{itemize}
	\item[-] L'affichage, le prompt et la réponse seront affichées par écrit sur  la console.
	\item[-] Le fonctionnement du jeu est simple : la pierre gagne sur les ciseaux, les ciseaux gagnent sur le papier, le papier gagne sur la pierre, deux éléments identiques correspondent à une égalité.
	\item[-] Toutes les chaînes de caractères, \og pierre\fg{}, \og papier\fg{} et \og ciseaux\fg{} doivent toujours être entrées en minuscule, le joueur devra donc écrire correctement ces mots, sinon vous devrez lui demander de redéfinir son choix.
	\item[-] S'il y a égalité, vous devrez relancer automatiquement votre programme (en regénérant un nouveau choix aléatoire pour la nouvelle session de jeu), jusqu'à ce qu'il y ait un gagnant à la partie.
\end{itemize}
\medskip

\subsection*{Exemples}
\begin{itemize}
	\item[-] Le choix aléatoire donne \og pierre\fg{} et le joueur a choisi \og papier\fg{} -> \texttt{Vous avez gagné ! Le papier enveloppe la pierre}
	\item[-] Le choix aléatoire donne \og ciseaux\fg{} et le joueur a choisi \og papier\fg{} -> \texttt{Vous avez perdu ! Les ciseaux coupent le papier}
	\item[-] Le choix aléatoire donne \og pierre\fg{} et le joueur a choisi \og pierre\fg{} -> \texttt{Égalité ! Recommencez...}
\end{itemize}
\medskip

\section{Ma solution}
\begin{lstlisting}
from random import randint

BDD = {
    "element": ["papier", "pierre", "ciseaux"],
    "gagnant": ["10", "21", "02"],
    "phrase": ["Le papier enveloppe la pierre",
               "Les ciseaux coupent le papier",
               "La pierre casse les ciseaux"]
}

while True:
    joueur = input("pierre, papier ou ciseaux: ")
    if joueur in BDD["element"]:
        choix, joueur = (randint(0, 2),
                         BDD["element"].index(joueur))

        if choix != joueur: break
        print("Égalité, recommencez...")

    else: print("Faute de frappe !")

print(f'Vous avez {"gagné" if f"{choix}{joueur}" in BDD["g
agnant"] else "perdu"} ! {BDD["phrase"][choix + joueur - 1]
} !')
\end{lstlisting}
\medskip

\section{Explication de l'algorithme}
\subsection*{Contexte}
\begin{itemize}
	\item[-] Trois éléments : \og pierre\fg{}, \og papier\fg{} ou \og ciseaux\fg{}.
	\item[-] Un choix aléatoire fait par la machine et un joueur qui entre son choix au clavier.
\end{itemize}
\medskip

Il suffit donc de réfléchir à un algorithme sympathique pour présenter le code de manière élégante et éviter bien sûr les répétitions.
\medskip

\subsection*{Rangement des données}
Chaque élément est rangé dans cet ordre particulier dans la liste, afin de les faire correspondre en triade logique.
\begin{itemize}
	\item[\textbullet] papier = index(0)
	\item[\textbullet] pierre = index(1)
	\item[\textbullet] ciseaux = index(2)
\end{itemize}
\medskip

\subsection*{Algorithme pour un choix triangulaire}
Si on fait l'addition \texttt{0+1}, on obtient \texttt{1}, alors le jeu se fait entre \og papier\fg{} et \og pierre\fg{}. On cherche ensuite la phrase dans \texttt{phrase} en faisant juste un calcul grâce à la somme \texttt{-1} des deux éléments. Donc en index:  \texttt{1-0 = 0}, et on trouve donc la chaîne de caractères \texttt{"Le papier enveloppe la pierre"}.
\medskip

Si on fait l'addition \texttt{1+2}, on obtient \texttt{3}, alors le jeu se fait entre la \og pierre\fg{} et les \og ciseaux\fg{}. Donc en index, on obtient \texttt{3-1}, soit \texttt{2}. On trouve donc la chaîne de caractères \texttt{"La pierre casse les ciseaux"}.
\medskip

De la même façon, si on fait l'addition \texttt{2+0}, on obtient \texttt{2} et le jeu se fait entre les \og ciseaux\fg{} et le \og papier\fg{}. En index cela donne \texttt{2-1}, soit \texttt{1}, et on tombe sur la chaîne de caractères \texttt{"Les ciseaux coupent le papier"}.
\medskip

\subsection*{\textit{The Winner is...}}
Pour connaître qui gagne, il suffit de convertir en \textit{string} et de joindre les deux caractères d'index du choix et du joueur. Ainsi, \texttt{10} dans \texttt{gagnant} veut dire que le choix aléatoire donne la \og pierre\fg{} (index \texttt{1}) et que le joueur a saisi le \og papier\fg{} (index \texttt{0}). La \og pierre\fg{} contre le \og papier\fg{} fait donc gagner le joueur.
\medskip

On affiche ainsi \texttt{"gagné"} puis la phrase qui suit s'obtient grâce à l'index de la liste \texttt{phrase} de la \texttt{BDD}, calculée par l'addition des deux index: \texttt{1+(0-1)}, ce qui nous donne \texttt{0}, ce qui correspond à la chaîne de caractères \texttt{"Le papier enveloppe la pierre"}.
\medskip

De la même façon pour \texttt{21} et \texttt{02}, cela représente la combinatoire complète des choix gagnants pour le joueur par rapport au choix aléatoire.
\medskip

\subsection*{Conclusion}
On utilise le calcul de la somme \texttt{-1} qui renvoie un objet de type \texttt{int} et qui pemret l'association des deux chaînes de caractères (\textit{string}) pour connaître le gagnant.